    \documentclass[11pt,
        usenames, % allows access to some tikz colors
        dvipsnames % more colors: https://en.wikibooks.org/wiki/LaTeX/Colors
    ]{article}
    \usepackage{
        amsmath,
        amssymb,
        % fouriernc, % fourier font w/ new century book
        fancyhdr, % page styling
        lastpage, % footer fanciness
        hyperref, % various links
        setspace, % line spacing
        amsthm, % newtheorem and proof environment
        mathtools, % \Aboxed for boxing inside aligns, among others
        float, % Allow [H] figure env alignment
        enumerate, % Allow custom enumerate numbering
        graphicx, % allow includegraphics with more filetypes
        wasysym, % \smiley!
        upgreek, % \upmu for \mum macro
        listings, % writing TrueType fonts and including code prettily
        tikz, % drawing things
        booktabs, % \bottomrule instead of hline apparently
        xcolor, % colored text
        cancel % can cancel things out!
    }
    \usepackage[margin=1in]{geometry} % page geometry
    \usepackage[
        labelfont=bf, % caption names are labeled in bold
        font=scriptsize % smaller font for captions
    ]{caption}
    \usepackage[font=scriptsize]{subcaption} % subfigures

    \newcommand*{\scinot}[2]{#1\times10^{#2}}
    \newcommand*{\dotp}[2]{\left<#1\,\middle|\,#2\right>}
    \newcommand*{\rd}[2]{\frac{\mathrm{d}#1}{\mathrm{d}#2}}
    \newcommand*{\pd}[2]{\frac{\partial#1}{\partial#2}}
    \newcommand*{\rdil}[2]{\mathrm{d}#1 / \mathrm{d}#2}
    \newcommand*{\pdil}[2]{\partial#1 / \partial#2}
    \newcommand*{\rtd}[2]{\frac{\mathrm{d}^2#1}{\mathrm{d}#2^2}}
    \newcommand*{\ptd}[2]{\frac{\partial^2 #1}{\partial#2^2}}
    \newcommand*{\md}[2]{\frac{\mathrm{D}#1}{\mathrm{D}#2}}
    \newcommand*{\pvec}[1]{\vec{#1}^{\,\prime}}
    \newcommand*{\svec}[1]{\vec{#1}\;\!}
    \newcommand*{\bm}[1]{\boldsymbol{\mathbf{#1}}}
    \newcommand*{\uv}[1]{\hat{\bm{#1}}}
    \newcommand*{\ang}[0]{\;\text{\AA}}
    \newcommand*{\mum}[0]{\;\upmu \mathrm{m}}
    \newcommand*{\at}[1]{\left.#1\right|}
    \newcommand*{\bra}[1]{\left<#1\right|}
    \newcommand*{\ket}[1]{\left|#1\right>}
    \newcommand*{\abs}[1]{\left|#1\right|}
    \newcommand*{\ev}[1]{\langle#1\rangle}
    \newcommand*{\p}[1]{\left(#1\right)}
    \newcommand*{\s}[1]{\left[#1\right]}
    \newcommand*{\z}[1]{\left\{#1\right\}}

    \newtheorem{theorem}{Theorem}[section]

    \let\Re\undefined
    \let\Im\undefined
    \DeclareMathOperator{\Res}{Res}
    \DeclareMathOperator{\Re}{Re}
    \DeclareMathOperator{\Im}{Im}
    \DeclareMathOperator{\Log}{Log}
    \DeclareMathOperator{\Arg}{Arg}
    \DeclareMathOperator{\Tr}{Tr}
    \DeclareMathOperator{\E}{E}
    \DeclareMathOperator{\Var}{Var}
    \DeclareMathOperator*{\argmin}{argmin}
    \DeclareMathOperator*{\argmax}{argmax}
    \DeclareMathOperator{\sgn}{sgn}
    \DeclareMathOperator{\diag}{diag\;}

    \colorlet{Corr}{red}

    % \everymath{\displaystyle} % biggify limits of inline sums and integrals
    \tikzstyle{circ} % usage: \node[circ, placement] (label) {text};
        = [draw, circle, fill=white, node distance=3cm, minimum height=2em]
    \definecolor{commentgreen}{rgb}{0,0.6,0}
    \lstset{
        basicstyle=\ttfamily\footnotesize,
        frame=single,
        numbers=left,
        showstringspaces=false,
        keywordstyle=\color{blue},
        stringstyle=\color{purple},
        commentstyle=\color{commentgreen},
        morecomment=[l][\color{magenta}]{\#}
    }

\begin{document}

\def\Snospace~{\S{}} % hack to remove the space left after autorefs
\renewcommand*{\sectionautorefname}{\Snospace}
\renewcommand*{\appendixautorefname}{\Snospace}
\renewcommand*{\figureautorefname}{Fig.}
\renewcommand*{\equationautorefname}{Eq.}
\renewcommand*{\tableautorefname}{Tab.}

\onehalfspacing

\pagestyle{fancy}
\rfoot{Yubo Su}
\rhead{}
\cfoot{\thepage/\pageref{LastPage}}

\title{Notes}
\author{Yubo Su}
\date{Date}

\maketitle

I'm trying a new format for research notes. I'm just going to go
chronologically, and never delete anything. So there will probably be a lot of
unclear/wrong stuff at the beginning, but that's the case with all my research
notes anyway.

\tableofcontents

\clearpage

\section{09/1/20---The Kuramoto Model} % probably wrong date

This is mostly review of the Kuramoto model as we learned in class in MATH 6270,
but where I actually try to do the algebra.

\subsection{Basic, Nonlinear Theory}

In the Kuramoto model, we consider $N$ phase-coupled oscillators
\begin{align}
    \dot{\theta}_i &= \omega_i + \sum\limits_{j \neq i}^N
            \frac{K}{N}\sin\p{\theta_j - \theta_i},\\
        &= \omega_i + \Re\s{\frac{K}{N}\sum\limits_{j \neq i}^N
            \frac{1}{i}e^{i\p{\theta_j - \theta_i}}}.
\end{align}
For a bit of an interlude, we define the complex variable $x_i = e^{i\theta_i}$
and obtain
\begin{align}
    \dot{x}_i &= ie^{i\theta_i} \rd{\theta_i}{t},\\
        &= ix_i \Re\s{\omega_i - \frac{iK}{N}\sum\limits_{j \neq i}^N
            x_jx_i^*},\\
        &= ix_i \Re\s{x_i^*\p{\omega_i x_i - \frac{iK}{N}
            \sum\limits_{j \neq i}^N x_j}}.
\end{align}
We have intentionally rearranged the terms a little bit for a bit more insight.
If the $\Re$ is dropped above, the EOM becomes completely linear, of form
$\dot{x} = Ax$. However, as it stands, $\dot{x}_i$ is always $\pi/2$ out of
phase with $x_i$, so the magnitude of the $x_i$ do not change. So the Kuramoto
model can be thought of as $N$ linearly coupled complex variables whose
magnitudes are then constrained to be fixed; that's the origin of the
nonlinearity.

For now, let's study the basic $\theta_i$ variable. Consider mean field
variables $re^{i\psi} = \ev{e^{i\theta}}$, then
\begin{equation}
    \dot{\theta}_i = \omega_i + \Im\s{K e^{-i\theta_i} re^{i\psi}}.
\end{equation}
We seek a steady state solution. Assume then $r$ is constant and $\psi = \Omega
t$, then we can always choose a corotating frame of reference that $\psi = 0$
(define $\theta_i + \Omega t \equiv \theta_i'$ and $\omega_i + \Omega \equiv
\omega_i'$ and drop the primes), which gives
\begin{equation}
    \dot{\theta}_i = \omega_i - Kr\sin \theta_i.
\end{equation}
This is the EOM Kuramoto analyzed.

Now, for a given $\theta_i$, if $\abs{\omega_i} > Kr$ then it will have no
equilibria, while if $\abs{\omega_i} < Kr$ then there will be fixed points where
$\omega_i - Kr\sin \theta_i = 0$; we call these \emph{drifters} and
\emph{locked} oscillators respectively. Can we find a self-consistent solution
for $r$ now? We assume the $\omega_i$ are drawn from some symmetric distribution
$g(\omega) = g(-\omega)$. Then the problem is symmetric for $\theta
\leftrightarrow -\theta$, so $\Im\s{\ev{e^{i\theta_i}}} = 0$ (this is
expected/necessary, since we are in the corotating frame where $\psi = 0$). For
the real parts:
\begin{itemize}
    \item For the drifting oscillators, they spend more time in some parts of
        the unit circle than others, so we think to measure their contribution
        in a time-averaged sense. This isn't strictly valid (time averages and
        ensemble averages are only equal in the ergodic limit), so an
        alternative way of stating this is that the drifting oscillators form a
        stationary distribution over the unit circle.

        In this approximation, the distribution $\rho(\theta, \omega) \propto 1
        / \abs{\theta_i}$, and is given w/ normalization (my notes)
        \begin{equation}
            \rho\p{\theta, \omega} = \frac{1}{2\pi}
                \frac{\sqrt{\omega^2 - (Kr)^2}}{\abs{\omega - Kr\sin \theta}}.
        \end{equation}
        Then, $\int\limits_0^{2\pi} \cos \theta \rho(\theta,
        \omega)\;\mathrm{d}\theta = 0$ because every contribution at $\theta$
        gets cancelled by the contribution at $\theta + \pi$.

    \item For the locked oscillators, they are locked where $\sin \theta_i =
        \omega_i / Kr$, so the distribution
        \begin{equation}
            f(\theta) = g(\omega) \rd{\omega}{\theta} = g\p{Kr \sin \theta}
                Kr \cos \theta.
        \end{equation}
\end{itemize}
The value of $r$ is then just the integral over the real part of the locked
oscillators
\begin{align}
    r &= \int\limits_{-Kr}^{Kr} g(\omega) \cos(\theta)\;\mathrm{d}\omega,\\
        &= \int\limits_{-\pi/2}^{\pi/2}
            Kr\cos^2\theta g\p{Kr\sin \theta}\;\mathrm{d}\theta.
            \label{eq:kuramoto_r_eq}
\end{align}

Of course $r = 0$ is a trivial solution; we want to know under what conditions a
solution appears for $r > 0$, so we assume $r$ is small. Then the simplest thing
to do is $g\p{Kr\sin\theta} \approx g(0)$, and we obtain
\begin{align}
    1 &\approx \int\limits_{-\pi/2}^{\pi/2}
        K\cos^2\theta g(0)\;\mathrm{d}\theta,\\
        &= \frac{\pi K g(0)}{2}.
\end{align}
Thus, when $K = 2 / (\pi g(0))$, we have a small, positive solution for $r$.

In fact, $K_c \equiv 2 / (\pi g(0))$ is the onset of collective synchronization,
i.e.\ for any $K > K_c$, there are solutions $r > 0$. We can't really see this
in our calculation above though, so we expand Eq.~\eqref{eq:kuramoto_r_eq}
to quadratic order in $r$ (the linear term vanishes since $g$ is even)
\begin{align}
    1 &= \int\limits_{-\pi/2}^{\pi/2}
        K\cos^2\theta \p{g(0) + \frac{g''(0)}{2}\p{Kr\sin\theta}^2}\;\mathrm{d}
            \theta,\\
        &= \frac{\pi K g(0)}{2}
            + \frac{g''(0)}{2}K^3r^2
            \int\limits_{-\pi/2}^{\pi/2}\cos^2\theta\sin^2\theta
                \;\mathrm{d}\theta,\\
        &= \frac{\pi K g(0)}{2}
            + \frac{g''(0)}{8}K^3r^2
            \int\limits_{-\pi/2}^{\pi/2}\sin^2\p{2\theta}
                \;\mathrm{d}\theta,\\
        &= \frac{\pi K g(0)}{2}
            + \frac{g''(0)}{8}K^3r^2
            \int\limits_{-\pi/2}^{\pi/2}\frac{1 + \cos\p{2\theta}}{2}
                \;\mathrm{d}\theta,\\
        &= \frac{\pi K g(0)}{2}
            + \pi\frac{g''(0)}{16}K^3r^2,\\
    r^2 &= \frac{16 - 8\pi K g(0)}{\pi K^3 g''(0)},\\
        &= -\frac{8g(0)}{K^2g''(0)}\frac{K - K_c}{K}.
\end{align}
The dimensions check out, as $g''K^2$ is dimensionless. Then if $g'' < 0$, the
case for most distributions, we see that solutions for $r$ only exist for $K >
K_c$, as expected. This is a supercritical bifurcation.

In conclusion, for the standard $g(\omega)$, a synchronized solution $r > 0$
spontaneously appears for $K > K_c$. The stability analysis is not too hard for
$r = 0$ and is very hard for $r > 0$, so I'll push off on it for the time being,
but the result is that $r = 0$ is linearly neutrally stable for $K < K_c$,
unstable for $K > K_c$, while $r  > 0$ is locally stable.

\section{09/21/20---Laplace Lagrange Secular Perturbation Theory}

In this section, we learn the derivation of the Laplace-Lagrange results that
couple the complex eccentricity and inclination vectors (see e.g.\ Pu \& Lai
\emph{Eccentricities and inclinations of multiplanet systems with external
perturbers}, though we will follow Murray \& Dermott).

\subsection{Leading Order Laplace-Lagrange Solution}

The idea behind Lagrange's planetary equations (M\&D \S6.8) is to rewrite the
potential in terms of the orbital elements (M\&D use $R = -\Phi$ the
\emph{disturbing function}), then the EOM are related to derivatives of $\Phi$.
If $\Phi$ does not explicitly depend on $\lambda$ the mean longitude, then $a$
is constant as well, and the EOM for the remaining for orbital elements are
\begin{subequations}\label{se:lagrange_planetary}
    \begin{align}
        \rd{e}{t} &= \frac{\sqrt{1 - e^2}}{na^2e}\pd{\Phi}{\varpi}
            &&\approx \frac{1}{na^2e}\pd{\Phi}{\varpi},\\
        \rd{\Omega}{t} &= -\frac{1}{na^2\sqrt{1 - e^2}\sin I}\pd{\Phi}{I}
            &&\approx -\frac{1}{na^2 I}\pd{\Phi}{I},\\
        \rd{\varpi}{t} &= -\frac{\sqrt{1 - e^2}}{na^2e}\pd{\Phi}{e}
            - \frac{\tan (I/2)}{na^2\sqrt{1 - e^2}}\pd{\Phi}{I},
            &&\approx -\frac{1}{na^2e}\pd{\Phi}{e}\\
        \rd{I}{t} &= \frac{\tan(I/2)}{na^2\sqrt{1 - e^2}}\pd{\Phi}{\varpi}
            + \frac{1}{na^2\sqrt{1 - e^2}\sin I}\pd{\Phi}{\Omega}
            &&\approx \frac{1}{na^2I}\pd{\Phi}{\Omega}.
    \end{align}
\end{subequations}
We include the leading order approximations for small $e, I$ as well. Recall $n
= \sqrt{G(M_\star + m) / a^3}$ is the mean motion.

We then grab the disturbing function between two planets from M\&D \S7.2. We
assume $GM_\star \approx n_i^2a_i^3$ for the $i$th planet, and expand to second
order in eccentricities and inclinations and to leading order in the masses,
then apply Lagrange's planetary equations. It's convenient to define the
components of the eccentricity and inclination vectors
\begin{align}
    h_j &= e_j \sin \varpi_j & k_j &= e_j \cos \varpi_j,\\
    p_j &= I_j \sin \Omega_j & q_j &= I_j \cos \Omega_j.
\end{align}
The equations for the evolution of these components is
\begin{align}
    \dot{h}_i &= \sum\limits_{j} A_{ij}k_j &
    \dot{k}_i &= \sum\limits_j -A_{ij}h_j,\\
    \dot{p}_i &= \sum\limits_j B_{ij}q_j &
    \dot{q}_i &= \sum\limits_j -B_{ij}p_j,
\end{align}
where
\begin{align}
    A_{ii} &= +n_i \sum\limits_{j \neq i}\frac{m_j}{M_\star + m_i}
        \frac{a_>}{a_<} b_{3/2}^{(1)}\p{\alpha_{ij}},&
    A_{ij} &= -n_i \frac{m_j}{M_\star + m_i}
        \frac{a_>}{a_<} b_{3/2}^{(2)}\p{\alpha_{ij}},\\
    B_{ii} &= -n_i\sum\limits_{j \neq i} \frac{m_j}{M_\star + m_i}
        \frac{a_>}{a_<} b_{3/2}^{(1)}\p{\alpha_{ij}},&
    B_{ij} &= +n_i \frac{m_j}{M_\star + m_i}
        \frac{a_>}{a_<} b_{3/2}^{(1)}\p{\alpha_{ij}},
\end{align}
where $\alpha_{ij} = a_< / a_>$, and we follow PL18's convention for the Laplace
coefficients. A useful approximation for $\alpha \ll 1$ is
\begin{align}
    b_{3/2}^{(n)}(\alpha) &\equiv \frac{1}{2\pi}\int\limits_0^\pi
        \frac{\cos (nt)}{(\alpha^2 + 1 - 2\alpha \cos t)^{3/2}}\;\mathrm{d}t,\\
    b_{3/2}^{(1)}(\alpha) &\approx \frac{3\alpha}{4} + \frac{43\alpha^3}{32}
        + \frac{525\alpha^5}{256} + \dots,\\
    b_{3/2}^{(2)}(\alpha) &\approx \frac{15\alpha^2}{16} +
        \frac{105\alpha^4}{64},
\end{align}
while when $\alpha \to 1$ the Laplace coefficients diverge such that $
b_{3/2}^{(2)} / b_{3/2}^{(1)} \to 1$. To be more precise, we note from M\&D that
(using our convention for the factor of $4$)
\begin{align}
    b_s^{(j)} &= 2\binom{s + j - 1}{j} \alpha^j F\p{s,s+j,j+1;\alpha^2},
\end{align}
where $F$ is the standard hypergeometric function. According to Wikipedia (and
Abramowitz \& Stegun apparently), the singular solution as $\alpha^2 \to 1$ goes
like $\p{1 - \alpha^2}^{(j + 1) - s - (s + j)} = \p{1 - \alpha^2}^{1 - 2s}$,
which is borne out by my numerical check, so all of the $b_{3/2}^{(n)}$ grow
like $\p{1 - \alpha}^{-2}$.

A very easy choice is to define the complex eccentricies and inclinations such
that $\mathcal{E}_i = e \exp\p{i\varpi}$ and $\mathcal{I} = I\exp\p{i\Omega}$,
so that
\begin{align}
    \dot{\mathcal{E}} &= i\bm{A}\mathcal{E}, &
    \dot{\mathcal{I}} &= i\bm{B} \mathcal{I}.
\end{align}
This is the general form of PL18, \S2.3 Eq.~(23), except they choose slightly
more symmetric form for the matrix elements
\begin{align}
    A_{ii} &= \sum\limits_{j \neq i} \frac{Gm_im_ja_<}{a_>^2 L_i}
        b_{3/2}^{(1)}(\alpha_{ij}),&
    A_{ij} &= -\frac{Gm_im_ja_<}{a_>^2 L_i}
        b_{3/2}^{(2)}(\alpha_{ij}),\\
    B_{ii} &= -\sum\limits_{j \neq i} \frac{Gm_im_ja_<}{a_>^2 L_i}
        b_{3/2}^{(1)}(\alpha_{ij}),&
    B_{ij} &= +\frac{Gm_im_ja_<}{a_>^2 L_i}
        b_{3/2}^{(1)}(\alpha_{ij}),
\end{align}
where $L_i \approx m_i\sqrt{GM_\star a_i}$ is the angular momentum. It bears
noting that $A_{ij}L_i = A_{ji}L_j$ and similarly for $B_{ij}$.

We can write down the general solution in terms of the eigenvectors $v_n$ and
eigenvalues $\lambda_n$ of $\bm{A}$, given as follows (unfortunately, $\bm{A}$
is not symmetric, so we don't have many guarantees on $v_n$ and $\lambda_n$):
\begin{equation}
    \mathcal{E}(t) = \sum\limits_{n} A_n v_n \exp\p{i\lambda_n t},
\end{equation}
where the coefficients $A_n$ are obtained by matching to initial conditions. Of
course, we can do similarly for $\mathcal{I}$. Note that the eigenvectors $v_n$
are real, so the $A_n$ are real as well. This solution corresponds to the free
eccentricity that is sometimes talked about in M\&D.

What are we interested in? We wonder whether there are are any situations in
which the eccentricity vectors spontaneously align, which means clustering in
$\varpi$. To linear order, it is obvious this is not possible: the only way the
phases of all the individual components of $\mathcal{E}$ vary in a synchronized
fashion is when the ICs are tuned such that the only $A_n$ that are nonzero are
those that have identical $\lambda_n$.

\subsection{Disk Expansion?}

What if we try naively to take the limit for a swarm of particles in a disk that
are initially coplanar? Assume a disk over some small extent $a \in [1 - \Delta
a / 2, 1 + \Delta a / 2]$ where $\Delta a \ll 1$, with total mass $M_d$.
Partition it into rings of width $\delta a$ such that, for simplicity, each ring
has mass $\sim M_d\frac{\Delta a}{\Delta a}$. Furthermore, the smallest
$\alpha_{ij}$ has $1 - \alpha_{ij} \approx \delta a$, so $1 - \alpha_{ij}^2
\approx 2\delta a$. Then, $b_{3/2}^{(n)}\p{\alpha_{ij}} \propto \p{\delta
a}^{-2}$, and the matrix elements $A_{ij}, B_{ij}$ still diverge like $\p{\delta
a}^{-1}$ due to the $L_i^{-1}$ term.

\subsection{Connection to Kuramoto}

What sorts of nonlinear restrictions would give a possibility of a Kuramoto-like
spontaneous synchronization? Recall our complexified Kuramoto model
\begin{align}
    \dot{x}_i &= ix_i \Re\s{x_i^*\p{\omega_i x_i - \frac{iK}{N}
            \sum\limits_{j \neq i}^N x_j}}.
\end{align}
Note that if we ignore taking the real part above, we obtain something that
looks like $\dot{x}_i = i\bm{M} x_i$, though $\bm{M}$ is complex and symmetric.
The effect of taking the real part is to project $\dot{x}_i$ to be out of phase
with $x_i$, such that the magnitude of $x_i$ never changes. We can imagine that
such a similar restriction to the Laplace-Lagrange solution can cause
synchronization, perhaps effected via the higher-order terms neglected in this
solution.

\section{09/22/20---More Test Applications of LL}

\subsection{Secular Disk}

The problem with our disk expansion above is obvious in hindsight: the secular
approximation breaks down if the rates of change $\gg n$. In practice, the
secular approximation will only work if $n$ grows in a way that the secular
timescale is constant, i.e.\ the matricies $\bm{A}, \bm{B}$ are everywhere
finite. This is obviously most strict on the diagonal terms, where there are $N
= \Delta a / \delta a$ terms in the summand while each term diverges like
$(\delta a)^{-1}$ as well. To keep the diagonal terms finite then, we need
$M_{\star} \propto N^4$. More precisely, choose all of the $m_i
\approx M / N$, write $a_< \approx a_> \approx a_0$ for simplicity, $L_i \approx
M / N\sqrt{G\p{M_\star N^{4}}a_0}$, then we obtain (we really should have used
the original forms instead of the PL18 ones, they work better)
\begin{align}
    A_{ij} \approx -B_{ij} &\propto
        -\frac{\sqrt{G}\p{M/N} \p{1 - \frac{\abs{j - i}}{N}\frac{\Delta a}{a_0}}}
        {N^2\sqrt{M_\star a_0^{3/2}}} \frac{1}{N^2},\\
        &\propto -\frac{nM/M_\star}{N}
        \p{1 - \frac{\abs{j - i}}{N}\frac{\Delta a}{a_0}},\\
    A_{ii} \approx -B_{ii} &\propto \frac{\sqrt{G}\p{M/N}}{N^2\sqrt{M_\star
        a_0^{3/2}}} \frac{1}{N^2}
        \sum\limits_{j \neq i}\p{1 - \frac{\abs{j - i}}{N}\frac{\Delta a}{a_0}},\\
        &\propto n\p{M / M_\star}
        \p{1 - \frac{i^2 + (N - i)^2}{N^2}\frac{\Delta a}{a_0}}.
\end{align}
The basic structure is a matrix whose diagonal elements have a slight dip in the
middle, and whose off-diagonal elements decrease going away from the diagonal
but depend only on $\abs{j - i}$. However, they don't fall off very quickly,
only linearly. Nevertheless, it is now possible, in theory, to get a solution
for all of the rings.

We note however a substantial discrepancy between this coupling and the Kuramoto
coupling: the off diagonal terms are \emph{real}, while in the Kuramoto model
they are imaginary. This comes because the oscillators couple via a sine in the
Kuramoto model, and via a cosine in the LL secular solution, as $\dot{\varpi}
\propto \pdil{\Phi}{e}$ where for two planets [M\&D Eq.~(7.6)]
\begin{equation}
    \Phi_1 = -\frac{n_1^2a_1^2m_2}{M_\star + m_1}\s{
        \frac{\alpha_{12}^2 b_{3/2}^{(1)}e_1^2}{8}
        - \frac{\alpha_{12}^2 b_{3/2}^{(1)}I_1^2}{8}
        - \frac{\alpha_{12}^2 b_{3/2}^{(2)}e_1e_2}{4}
            \cos\p{\varpi_1 - \varpi_2}
        + \frac{\alpha_{12}^2 b_{3/2}^{(2)}I_1I_2}{4}
            \cos\p{\Omega_1 - \Omega_2}}.
\end{equation}
It seems like this may be a dead end then, unless we can find a disturbing
function $\propto \sin\p{\Delta \varpi}$. This is somewhat surprising though: if
we have two rings of mass, their mutual gravitational potential is singular if
they cross, but most singular if they overlap. This must be a consequence of the
multipole expansion: we aren't computing the interaction between two rings at
$a_1, a_2$, we are computing the interaction between two quadrupoles (and
octupoles, for $b_{3/2}^{(2)}$) located at the origin, which again is only
accurate if the system is hierarchical, $\alpha_{ij} \lesssim 1$. Our error lies
then in the multipole expansion of the disturbing function; maybe there is
another way.

\section{Brainstorming Other Approaches}

In yet another manner of thinking, the LL approach to this problem doesn't
produce the right dynamics since the self-gravity of the ring is a small
correction, and we only capture self-gravity-induced precession. If we want any
stronger self-gravity, it is very difficult to do with the secular equations of
motion above, since we always have to send $M_\star \to \infty$ to keep the
secular approximation valid. Instead, it seems we should fundamentally be doing
calculations with rings, and it is likely the gravitational interaction between
these rings must be softened somehow. This is somewhat reminiscent of plasma
physics and the collisionless Boltzmann equation. Another approximation we could
do is to still use the Lagrange planetary equations, but find a different way to
expand the interaction potential/disturbing function.

\end{document}

